
\chapter{For Software Developers}

\section{Code Style Guide}

This section defines a style guide which should be followed by all code that is written in \robotlib.  Being consistent with this style will make the code easier to read, debug, and maintain.  The section was inspired by the C++ style guide for ROS: \url{http://www.ros.org/wiki/CppStyleGuide}.  It makes use of the follow shortcuts for naming schemes:

\begin{itemize}
\item \mcode{CamelCased}: The name starts with a capital letter, and has a capital letter for each new word, with no underscores.
\item \mcode{camelCased}: Like CamelCase, but with a lower-case first letter
\item \mcode{under_scored}: The name uses only lower-case letters, with words separated by underscores.
\item \mcode{Under_scored}: The name starts with a capital letter, then uses \mcode{under_score}.
\item \mcode{ALL_CAPITALS}: All capital letters, with words separated by underscores.
\end{itemize}

\noindent Note: Some of the files in the repository were written before this style guide.  
If you find one, rather than trying to change it yourself, log a bug in bugzilla.


\begin{itemize}
\item In General: 
\begin{itemize}
\item Robot Names are \mcode{CamelCased}.
\end{itemize}
\item In Java:
\begin{itemize}
\item Class names (and therefore class filenames/directories) are \mcode{CamelCased}
\item  Methods names are \mcode{camelCased}
\item  Variable names are \mcode{under_scored}
\item Member variables are \mcode{under_scored} with a leading \mcode{m_} added
\item Global variables are \mcode{under_scored} with a leading \mcode{g_} added
\item  Constants are \mcode{ALL_CAPITALS}
\item Every class and method should have a brief "javadoc" associated with it.
\item  All java classes should be in packages relative to the locomotion svn root, e.g.: \\
   \mcode{package robotlib.examples.Pendulum;} \\
   \mcode{package robots.compassTripod;}
\end{itemize}
\item In \matlab: 
\begin{itemize}
\item All of the above rules hold, except:
\item Member variables need not start with \mcode{m_} since the requirement that they are referenced with obj.var makes the distinction from local variables clear
\item Variable names that describe a matrix (instead of vector/scalar) are \mcode{Under_scored}.
\item Calls to \matlab class member functions use \mcode{obj = memberFunc(obj,...) }.
\item  All methods begin by checking their inputs (e.g. with \mcode{typecheck.m}).
\end{itemize}
\item In C++: 
\begin{itemize}
\item All of the above rules still hold.
\item Filenames for \mcode{.cpp} and \mcode{.h} files which define a single class are \mcode{CamelCased}.
\item Filenames for \mcode{.cpp} and \mcode{.h} files which define a single method are \mcode{camelCased}.  
\item Filenames for any other \mcode{.cpp} and \mcode{.h} files are \mcode{under_scored}.
\end{itemize}
\item In LCM:
\begin{itemize}
\item LCM types are \mcode{under_scored} with a leading \mcode{lcmt_} added. If the type is specific to a particular robot, then it begins with \mcode{lcmt_robotname_}.
\item Channel names are \mcode{under_scored}, and ALWAYS begin with \mcode{robotname_}. 
  \emph{Although robotnames are \mcode{CamelCased}, their use in LCM channels and types should be all lowercase}
\item Variable names in LCM types follow the rules above.
\end{itemize}
%\item In Makefiles: 
 %  garratt says use CMAKE??
\end{itemize}


\section{Check-In Procedures}

This section defines the requirements that must be met before anything 
is committed to the main branch (\mcode{trunk}) of the \robotlib repository.

\subsection{Unit tests}

  Whenever possible, add test files (in any subdirectory test) any code 
  that you have added or modified.  These take a little time initially, 
  but can save incredible amounts of time later.

\subsection{Run all tests}

  Before committing anything to the repository, the code must pass all of 
  the unit tests.  Use the following script to check:
     \mcode{robotlib/runAllTests.m} 

\subsection{Matlab Reports}
  There are a number of helpful matlab reports, that can be run using: 
    \mcode{Desktop>Current Directory},
  then 
    \mcode{Action>Reports}  (the Action menu is the gear icon)

  Before a commit, your code should pass the following reports:
\begin{itemize}
 \item Contents report
  \item  Help report  (with all but the Copyright options checked)
\end{itemize}
  and you should run the M-Link Code Check report to look for useful suggestions.

\subsection{Contributing Code}

If you don't have write permissions to the repository, then please make sure that your changes meet the requirements above, then email a patch to Russ by running 
\begin{lstlisting}
svn diff > my_patch.diff
\end{lstlisting}
in your main \robotlib directory, then email the diff file.

\section{Version Number}

Version number has format W.X.Y.Z where
\begin{itemize}
\item W= major release number
\item X = minor release number
\item Y = development stage*
\item Z = build
\end{itemize}
* Development stage is one of four values:
\begin{itemize}
\item 0 = alpha (buggy, not for use)
\item 1 = beta (mostly bug-free, needs more testing)
\item 2 = release candidate (rc) (stable)
\item 3 = release
\end{itemize}
Z (build) is optional. This is probably not needed but could just refer to the revision of the repo at the time of snapshot. Numbered versions should be referenced via tags.
