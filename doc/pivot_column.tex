\documentclass{article}
\newtheorem{theorem}{Theorem}
\newtheorem{lemma}{Lemma}
% necessary packages
\usepackage{amsmath}
\usepackage{amssymb}
\usepackage{bm}
\usepackage{oldstyle}
\usepackage{color}
\usepackage{soul}
% disabled b/c we're using tcolorbox
\usepackage[usenames,dvipsnames,svgnames,table]{xcolor}

% math stuff
\providecommand{\inv}[1]{{#1}^{\ensuremath{\mathsf{-1}}}} % inverse ...
\providecommand{\tr}[1]{{#1}^{\ensuremath{\mathsf{T}}}} % transpose ...
\providecommand{\itr}[1]{{#1}^{\ensuremath{\mathsf{-T}}}} % transpose ...
\providecommand{\strans}[1]{{#1}^{\ensuremath{\mathsf{S}}}} % transpose ...
\providecommand{\abs}[1]{\lvert#1\rvert}
\providecommand{\norm}[1]{\left\lVert#1\right\rVert}
\providecommand{\svect}[1]{\hat{\bm{#1}}}
\providecommand{\vect}[1]{\bm#1}
\providecommand{\mat}[1]{\mathbf#1}
\providecommand{\smat}[1]{\hat{\mathbf{#1}}}
\providecommand{\bigTheta}{\mathit{\Theta}}
\providecommand{\bigO}{\mathit{O}}
\providecommand{\sgn}[1]{\textrm{sgn}(#1)}
\providecommand{\xform}[3]{\hspace{0mm}_#1\mat{#2}_#3}
\newcommand{\ud}{\,\mathrm{d}}

% lists
\newcommand{\ia}{(\textos{a})\!}
\newcommand{\ib}{(\textos{b})\!}
\newcommand{\ic}{(\textos{c})\!}
\newcommand{\id}{(\textos{d})\!}
\newcommand{\ie}{(\textos{e})\!}
\newcommand{\1}{(\textos{1})\!}
\newcommand{\2}{(\textos{2})\!}
\newcommand{\3}{(\textos{3})\!}
\newcommand{\4}{(\textos{4})\!}
\newcommand{\5}{(\textos{5})\!}
\newcommand{\6}{(\textos{6})\!}
\newcommand{\7}{(\textos{7})\!}
\newcommand{\8}{(\textos{8})\!}

\DeclareMathOperator*{\minimize}{minimize}

% indicate necessary changes
\providecommand{\captiontxt}[1]{\caption{\emph{#1}}}
\providecommand{\note}[1]{\textcolor{orange}{#1}} 
\providecommand{\changeme}[1]{\textcolor{red}{#1}} 
\providecommand{\TODO}[1]{
\sethlcolor{yellow}
\noindent
$\cdot$\ \hl{#1}
\\
} 

\title{Computing the principal pivoting transform for solving Linear Complementarity Problems with Lemke's Algorithm}
\date{}
\author{Hongkai Dai and Evan Drumwright\\Toyota Research Institute}

\begin{document}
\maketitle
\section{Problem statement} Given tuples $w = \{ w_1, \hdots, w_n \}$ and $z = \{z_1, \hdots, z_n, z_{n+1}\}$, which can be arranged into vectors as follows:
\begin{align*}
	\vect{w} =& \tr{\begin{bmatrix}w_1 & \hdots & w_n \end{bmatrix}}\in\mathbb{R}^n \\
		\vect{z} =& \tr{\begin{bmatrix}z_1 & \hdots & z_n & z_{n+1} \end{bmatrix}}\in\mathbb{R}^{n+1},
\end{align*}
we seek to permute the relationship:
\begin{align*}
	\vect{w} = \vect{q} + \mat{M}\vect{z}
\end{align*}
where $\vect{q}\in\mathbb{R}^n, \mat{M}\in\mathbb{R}^{n\times (n+1)}$ are a given vector/matrix (note that boldface is used to help distinguish vectors and matrices from sets). $z_{n+1}$ will hereforth be referred to as the ``artificial variable'' by swapping some entries in the ``dependent variable tuple'' $w$ with some entries in the ``independent variable'' tuple $z$, yielding two new sets $z'$ and $w'$. Similarly to the matrix/vector relationship above, vectors $\vect{z}' \in \mathbb{R}^{n+1}$ and $\vect{w}' \in \mathbb{R}^n$ can be defined that correspond to variable indices of $z$ and $w$. \emph{The problem that this document seeks to solve is the vector} $\vect{q}'$ \emph{and the matrix} $\mat{M}'$ (actually, just a single column of it) \emph{such that the relationship}:
\begin{align*}
\vect{w}' = \vect{q}' + \mat{M}'\vect{z}'
\end{align*}
follows from the ordering in the vectors corresponding to the positions of the
individual $w$ and $z$ variables in the tuples $w'$ and $z'$. Recall that
pivoting takes the value $\vect{z'} = \vect{0}$, which allows $\vect{w'}$ to be
determined simply be setting it equal to $\vect{q}'$, via the equation above.
This rearrangement is known as a \emph{principal pivoting transform}. 

One of the variables in the independent variable tuple $z'$ will be identified as the \textit{driving variable}. We will denote this variable as $z'_{\textrm{driving}}$. Our goal is to find the vector $\vect{q}'$, together with the column in $\mat{M}'$ that will be multiplied (i.e., via the inner product operation) with $z'_{\textrm{driving}}$. We will denote this column as $\mat{M}'_{\textrm{driving}}$.

\subsubsection{Running example} 
We consider the LCP from Example 4.4.7 in \cite{Cottle:1992}:
\begin{align*}
	\begin{bmatrix} w_1 \\ w_2 \\ w_3\end{bmatrix} = \underbrace{\begin{bmatrix}-3 \\ 6 \\ -1\end{bmatrix}}_{q} + \underbrace{\begin{bmatrix}  0 & -1 & 2 & 1\\ 2 & 0 & -2 & 1 \\ -1 & 1 & 0 & 1\end{bmatrix}}_{M}\begin{bmatrix}z_1 \\ z_2 \\ z_3 \\ z_4\end{bmatrix}
\end{align*}
After some number of pivoting operations, assume that the ``dependent'' tuple $w'$ and ``independent'' tuple $z'$ consist of:
\begin{align}
	w' = \{ z_4, w_2, z_3 \} ,
	z' = \{ w_1, w_3, z_2, z_1 \} 
    \tag{1} % NOTE: Do not change equation number without changing
            % unrevised_lemke_solver_test.cc
\end{align}


\subsection{Terminology} \label{subsec:terminology}
We now introduce some terminology:
\begin{itemize}
	\item \textsc{independent w}: the tuple of variables from $w$ that (partially) comprise $z'$ in the order in which they are found in $w$  (i.e., the elements of \textsc{independent w} appear in ascending order, e.g., $\{ w_1, w_2 \}$). In the running example, \textsc{independent w} is $\{ w_1, w_3 \}$.
		\begin{itemize}
			\item  $\alpha$: the positions in $w$ of the elements from \textsc{independent w}. In the running example, $\alpha = \{ 1, 3 \}$. From the definition of \textsc{independent w}, these $w$ elements of $\alpha$ appear in ascending order.
			\item $\alpha'$: the respective indices in $z'$ of the variables from \textsc{independent w}. In the running example, $\alpha' = \{ 1, 2\}$. These $w$ elements of $\alpha'$ \emph{elements will not necessarily be present in ascending order.} 
		\end{itemize}
		Note that the elements in the ``vanilla'' tuple correspond to variable indices, while elements in the primed tuple correspond to tuple indices. Also, \textbf{observe} the invariant $w_{\alpha_i} = z'_{\alpha'_i}$.
	\item \textsc{dependent z}: the tuple of variables from $z$ that (partially) comprise $w'$ in the order in which they are found in $z$ (i.e., the elements of \textsc{dependent z} appear in ascending order, e.g., $\{ z_3, z_4 \}$). In the running example, \textsc{dependent z} is $\{ z_3, z_4 \}$.
		\begin{itemize}
			\item $\beta$: the positions in $z$ of the elements from \textsc{dependent z}. In the running example, $\beta = \{3, 4\}$ (corresponding to $z_3$ and $z_4$). From the definition of \textsc{dependent z}, these $z$ elements of $\beta$ appear in ascending order.
			\item $\beta'$: the respective indices in $w'$ of the variables from \textsc{dependent z}. In the running example, $\beta' = \{3, 1\}$. These $z$ elements of $\alpha'$ \emph{elements will not necessarily be present in ascending order.}
		\end{itemize}
		Again, note that the elements in the ``vanilla'' tuple correspond to variable indices, while elements in the primed tuple correspond to tuple indices. Also, \textbf{observe} the invariant $z_{\beta_i} = w'_{\beta'_i}$.
	\item \textsc{dependent w}: the tuple of variables from $w$ that (partially) comprise $w'$ in the order in which they are found in $w$ (i.e., the elements of \textsc{dependent w} appear in ascending order, e.g., $\{ w_1, w_2 \}$). In the running example, \textsc{dependent w} is $\{ w_2 \}$.
		\begin{itemize}
			\item $\bar{\alpha}$: the positions in $w$ of the elements from \textsc{dependent w}. In the running example, $\bar{\alpha} = \{ 2 \}$. From the definition of \textsc{dependent w}, these $w$ elements of $\bar{\alpha}$ appear in ascending order.
			\item $\bar{\alpha}'$: the respective indices in $w'$ of the variables from \textsc{dependent w}. In the running example, $\bar{\alpha}' = \{ 2 \}$. These $w$ elements of $\bar{\alpha}$'s \emph{elements are not necessarily present in ascending order}.
		\end{itemize}
		Again, note that the elements in the ``vanilla'' tuple correspond to variable indices, while elements in the primed tuple correspond to tuple indices. Also, \textbf{observe} the invariant $w_{\bar{\alpha}_i} = w'_{\bar{\alpha'}_i}$.
	\item \textsc{independent z}: the tuple of variables from $z$ that (partially) comprise $z'$ in the order in which they are found in $z$ (i.e., the elements of \textsc{independent z} appear in ascending order, e.g., $\{ z_1, z_2 \}$). In the running example, \textsc{independent z} is $\{ z_1, z_2 \}$. The elements in \textsc{independent z} are present in ascending order (e.g., $\{ z_1, z_2 \}$).
		\begin{itemize}
			\item $\bar{\beta}$: one greater, respectively, than the positions in $z$ of the elements from \textsc{independent z}. In the running examples, $\bar{\beta} = \{1, 2 \}$. From the definition of \textsc{independent z}, the elements of $\bar{\beta}$'s elements appear in ascending order.
			\item $\bar{\beta}'$: the respective indices in $z'$ of the variables from \textsc{independent z}. In the running example, $\bar{\beta}' = \{4, 3 \}$. These $z$ elments of $\bar{\beta}'$ \emph{elements are not necessarily present in ascending order}.
		\end{itemize}
		Once more, note that the elements in the ``vanilla'' tuple correspond to variable indices, while elements in the primed tuple correspond to tuple indices. Also, \textbf{observe} the invariant $z_{\bar{\beta}_i} = z'_{\bar{\beta}'_i}$.
\end{itemize}

\section{Approach}
\begin{lemma}
	The length of \textsc{independent w} is the same as that of \textsc{dependent z}.
\end{lemma}
We will denote the length of \textsc{independent w} as $m$. In the running example, $m = 2$.
\subsection{Computing $q'$}
We first define a \textit{square} matrix $\mat{M}^{\alpha\beta} \in\mathbb{R}^{m \times m}$ as being comprised of entries $i \in 1,\ldots,m$ and $j \in 1, \ldots, m$:
\begin{align*}
	\mat{M}^{\alpha\beta}_{ij} = \mat{M}_{\alpha_i\beta_j}
\end{align*}
and a rectangular matrix $\mat{M}^{\bar{\alpha}\beta} \in\mathbb{R}^{(n-m)\times m}$ as being comprised of entries $i \in 1,\ldots,n-m$ and $j \in 1, \ldots, m$:
\begin{align*}
	\mat{M}^{\bar{\alpha}\beta}_{ij} = \mat{M}_{\bar{\alpha}_i\beta_j}
\end{align*}

Likewise, we define vectors $\vect{q}^{\alpha} \in \mathbb{R}^m$ and $\vect{q}^{\bar{\alpha}} \in \mathbb{R}^{(n-m)}$ as being comprised of entries $i \in 1,\ldots,m$ and $j \in 1, \ldots, n-m$:
\begin{align*}
	q^{\alpha}_i & = q_{\alpha_i} \\
	q^{\bar{\alpha}}_j & = q_{\bar{\alpha}_j}
\end{align*}
In the running example---recall that $\alpha = \{1, 3\}, \beta = \{3, 4\},
\bar{\alpha} = \{2 \}$---we highlight the parts of $\vect{q}$ and $\mat{M}$ that correspond to $\textcolor{red}{q^{\alpha}}$, $\textcolor{red}{M^{\alpha\beta}}$ in red, and $\textcolor{blue}{q^{\bar{\alpha}}}$, $\textcolor{blue}{M^{\bar{\alpha}\beta}}$ in blue. Now putting the example into ``tableaux form'':
\begin{center}
	\begin{tabular}{|c|c|c c c c|}
		\hline
		      &  & $z_1$ & $z_2$ & $\textcolor{red}{z_3}$ & $\textcolor{red}{z_4}$\\
		\hline
		$\textcolor{red}{w_1}$ &\textcolor{red}{-3} & 0 & -1 & \textcolor{red}{2} & \textcolor{red}{1}\\
		$\textcolor{blue}{w_2}$ & \textcolor{blue}{6} & 2 & 0 & \textcolor{blue}{-2} & \textcolor{blue}{1}\\
		$\textcolor{red}{w_3}$ & \textcolor{red}{-1} & -1 & -1 & \textcolor{red}{0} & \textcolor{red}{1}\\
		\hline
	\end{tabular}
\end{center}
thereby yielding the following vectors and matrices:
\begin{align*}
		\textcolor{red}{q^{\alpha} = \begin{bmatrix} -3 \\ -1\end{bmatrix}},
		\textcolor{blue}{q^{\bar{\alpha}} = \begin{bmatrix} 6\end{bmatrix}},
		\textcolor{red}{M^{\alpha\beta} = \begin{bmatrix} 2 & 1 \\ 0 & 1\end{bmatrix}},
		\textcolor{blue}{M^{\bar{\alpha}\beta} = \begin{bmatrix} -2 & 1 \end{bmatrix}}
\end{align*}
With the index sets $\alpha', \bar{\alpha}'$ defined in Subsection
\ref{subsec:terminology}, and Equation 10 from Page 71 of~\cite{Cottle:1992}, we
can write $\vect{q}'$ as:
\begin{align}
	\vect{q}'^{\beta'} = -(\mat{M}^{\alpha\beta})^{-1}\vect{q}^{\alpha}
 \tag{2} \\ 
  % NOTE: Do not change equation number without changing
  % unrevised_lemke_solver.cc
	\vect{q}'^{\bar{\alpha}'} = \vect{q}^{\bar{\alpha}} +
    \mat{M}^{\bar{\alpha}\beta}\vect{q}'^{\beta'}
 \tag{3} 
  % NOTE: Do not change equation number without changing
  % unrevised_lemke_solver.cc
\end{align}
where $\vect{q}'^{\alpha'}, \vect{q}'^{\bar{\alpha}'}$ are ``views'' of the
vector $q'$, as $\vect{q}'^{\alpha'}_i = \vect{q}'_{\alpha'_i},
\vect{q}'^{\bar{\alpha}'}_i = \vect{q}'_{\bar{\alpha}'_i}$. In the running
example $\vect{q}'^{\alpha'} = \tr{\begin{bmatrix}1 & 1\end{bmatrix}},
\vect{q}'^{\bar{\alpha}'} = [5]$. \textbf{$\vect{q}'$ is not composed by
stacking $\vect{q}'^{\alpha'}, \vect{q}'^{\bar{\alpha}'}$ as is done in similar procedures in~\cite{Cottle:1992}.}



\subsection{Computing $\mat{M}'_{\textrm{driving}}$}
This computation requires considering two cases for the driving variable, which is an entry in the new independent variable $z'$. The driving variable can be either:
\begin{enumerate}
		\item a dependent variable from $w$.
		\item an independent variable $z$.
\end{enumerate}
in the running example, if the driving variable is $w_1$ or $w_3$, then it belongs to the first case; if the driving variable is $z_2$ or $z_1$, then it belongs to the second case. Let's discuss the two cases separately.

\subsubsection{Driving variable is a dependent variable from $w$}
If the driving variable $z'_{\textrm{driving}}$ is a dependent variable from $w$, then it also belongs to the vector \textsc{independent w}. We denote the index of $z'_{\textrm{driving}}$ in \textsc{independent w} as $\gamma$, namely: 
\begin{align}
	w^{\alpha}_\gamma = w_{\alpha_\gamma} = z'_{\textrm{driving}} \tag{4}
  % NOTE: Do not change equation number without changing
  % unrevised_lemke_solver.cc
\end{align}
where $w^{\alpha}$ is the vector \textsc{independent w}. In the running example,
\textsc{independent w} is $\{ w_1, w_3 \}$, thus $\alpha = \{ 1, 3 \}$. If
$z'_{\textrm{driving}} \equiv w_1$, then the index $\gamma \equiv 1$. If
$z'_{\textrm{driving}} \equiv w_3$, then the index $\gamma \equiv 2$. To compute
the column vector $\mat{M}'_{\textrm{driving}}$, we need to first compute the
$\gamma^{\textrm{th}}$ column of the matrix $(\mat{M}^{\alpha\beta})^{-1}$. To
this end, we define a unit vector $\vect{e} \in\mathbb{R}^m$ as:
\begin{align*}
	&e_{\gamma} = 1\\
	&e_i = 0 \quad \text{if } i \neq \gamma
\end{align*}
and we can compute $M'_{\textrm{driving}}$ as:
\begin{align}
	&\mat{M}'^{\beta'}_{\textrm{driving}} =
    (\mat{M}^{\alpha\beta})^{-1}\vect{e}\label{eq:b_prime_alpha_prime} 
    \tag{5}\\
  % NOTE: Do not change equation number without changing
  % unrevised_lemke_solver.cc
	&\mat{M}'^{\bar{\alpha}'}_{\textrm{driving}} =
    \mat{M}^{\bar{\alpha}\beta}\mat{M}'_{\textrm{driving}_{\beta'}} \tag{6}
  % NOTE: Do not change equation number without changing
  % unrevised_lemke_solver.cc
\end{align}
Notice that $\mat{M}'_{\textrm{driving}_{\alpha'}}$ is the
$\gamma^{\textrm{th}}$ column of $\mat{M}_{\alpha\beta}^{-1}$ according to Equation~\eqref{eq:b_prime_alpha_prime}.

\subsubsection{Driving variable is an independent variable from $z$}
If the driving variable $z'_{\textrm{driving}}$ is an independent variable from $z$, we denote the position in $z$ of $z'_{\textrm{driving}}$ as $\zeta$. In the running example, if the driving variable is $z_2$, then $\zeta \equiv 2$; if the driving variable is $z_1$, then $\zeta \equiv 1$.

To compute the column vector $\mat{M}'_{\textrm{driving}}$ in $\mat{M}'$, we
need the $\zeta^{\textrm{th}}$ column of $\mat{M}$, denoted $\vect{g}$, which
will be decomposed into two sub-vectors. One sub-vector  $\vect{g}^{\alpha}$
contains the entries with row indices in $\alpha$ while the other,  $\vect{g}^{\bar{\alpha}}$, contains the entries with row indices from $\bar{\alpha}$. Namely:
\begin{align*}
	&g^{\alpha}_i = M_{\alpha_i\zeta}\\
	&g^{\bar{\alpha}}_i = M_{\bar{\alpha}_i\zeta}
\end{align*}

In the running example, $\textcolor{red}{w_1, w_3}$ will be pivoted to $z'$, and
$\textcolor{blue}{w_2}$ will remain in $w'$. If the driving variable is $z_2$,
we highlight the vector $\textcolor{red}{g^{\alpha}},
\textcolor{blue}{g^{\bar{\alpha}}}$ in $\mat{M}$ as:
\begin{center}
\begin{tabular}{|c|cccc|}
	\hline
		& $z_1$ & $z_2$ & $z_3$ & $z_0$\\
		\hline
		$\textcolor{red}{w_1}$ & 0 & \textcolor{red}{-1} & 2 & 1\\
		$\textcolor{blue}{w_2}$ & 2 & \textcolor{blue}{0} & -2 & 1\\
		$\textcolor{red}{w_3}$ & -1 & \textcolor{red}{1} & 0 & 1\\
		\hline
\end{tabular}
\end{center}
so $\textcolor{red}{g^{\alpha}} \equiv \tr{\begin{bmatrix}-1 & 1\end{bmatrix}}, \textcolor{blue}{g^{\bar{\alpha}}} = [0]$.

with the vector $\vect{g}^{\alpha}, \vect{g}^{\bar{\alpha}}$, we can then
compute $\mat{M}'_{\textrm{driving}}$ as
\begin{align}
	&\mat{M}'^{\beta'}_{\textrm{driving}} =
    -(\mat{M}^{\alpha\beta})^{-1}\vect{g}^{\alpha} \tag{7} \\
  % NOTE: Do not change equation number without changing
  % unrevised_lemke_solver.cc
	&\mat{M}'^{\bar{\alpha}'}_{\textrm{driving}} = \vect{g}^{\bar{\alpha}} +
    \mat{M}^{\bar{\alpha}\beta}\mat{M}'^{\beta'}_{\textrm{driving}} \tag{8}
  % NOTE: Do not change equation number without changing
  % unrevised_lemke_solver.cc
\end{align}

\bibliographystyle{abbrv}
\begin{thebibliography}{56}
\bibitem{Cottle:1992}
Richard W. Cottle, Jong-Shi Pang, and R.E. Stone. \emph{The Linear
Complementarity Problem}, Academic Press, 1992.
\end{thebibliography}

\end{document}
