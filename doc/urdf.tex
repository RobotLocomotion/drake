
\chapter{Universal Robot Description Format (URDF)}

This appendix documents our support for the Universal Robot Description Format (URDF).

The primary documentation for the URDF specification is available on the ROS website:\\
\centerline{ \url{http://www.ros.org/wiki/urdf/XML}}

\smallskip \noindent
Drake supports most of the tags specified in this documentation, including the transmission elements which are documented seperately here\\ \centerline{\url{http://ros.org/wiki/urdf/XML/Transmission}}  

\smallskip
However, there is currently little or no support for the \mcode{<sensor>} elements.  We have added a few additional tags

\begin{itemize}
\item \mcode{<loop_joint>}
\begin{itemize}
\item Adds support for simple closed-loop kinematic chains.  For an example, see \mcode{examples/SimpleFourBar/FourBar.urdf}.  
\item Attributes
\begin{itemize}
\item \mcode{name} (required)
\item \mcode{type} (required). Currently must be 'continuous'.  The intent is to eventually support all of the same types as a \mcode{<joint>}. See the \mcode{<joint>} element documentation. 
\end{itemize}
\item Elements
\begin{itemize}
\item \mcode{<link1>} (required)
\begin{itemize}
\item Attributes: \mcode{link} (required). String naming the link on one side of the joint.
\item Elements: \mcode{<origin>} (optional). Location (and orientation) of joint relative to the coordinate system on link 1. Default is \mcode{xyz=[0 0 0]}, \mcode{rpy=[0 0 0]}. 
\end{itemize}
\item \mcode{<link2>} (required)
\begin{itemize}
\item Attributes: \mcode{link} (required)
\item Elements: \mcode{<origin>} (optional). Location (and orientation) of joint relative to the coordinate system on link 2. 
\end{itemize}
\item \mcode{<axis>} (optional) 
\item Note: We do not support dynamics or torque at the \mcode{<loop_joint>} yet. (see bug 921) 
\end{itemize}
\end{itemize}

\item \mcode{<force_element>}
\begin{itemize}
\item Adds dynamic, force-producing elements like springs and
  aerodynamics
\item Attributes
\begin{itemize}
\item \mcode{name} (required).
\item \mcode{type} (required).  Currently support includes
  \mcode{LinearSpringDamper, Wing, and Thrust}.  More types will be added as they are identified.
\end{itemize}

\item Elements for type \mcode{LinearSpringDamper}
\begin{itemize}
\item \mcode{<link1>} (required)  
\begin{itemize}
\item Attributes: \mcode{link} (required). String naming the link on
  one side of the joint; \mcode{<xyz>} (optional). Location of joint relative to the coordinate system on link 1. Default is \mcode{xyz=[0 0 0]}.
\end{itemize}
\item \mcode{<link2>} (required)
\begin{itemize}
\item Attributes: \mcode{link} (required). String naming the link on
  one side of the joint; \mcode{<xyz>} (optional). Location of joint relative to the coordinate system on link 2. Default is \mcode{xyz=[0 0 0]}.
\end{itemize}
\item \mcode{<rest_length>} (default 0)
\item \mcode{<stiffness>} in Newtons per meter (default 0)
\item \mcode{<damping>} in Newton seconds per meter.  positive values
  resist motion.  (default 0)
\end{itemize}

\item Elements for type \mcode{Wing}.  See RigidBodyWing.m for more documentation.
\begin{itemize}
\item \mcode{<parent>} (required). 
\begin{itemize} 
\item Attributes: \mcode{<link>} (required).  String naming
  the link on which this wing is attached.
\end{itemize}
\item \mcode{<origin>} (required) with attributes \mcode{xyz}.
  This defines the position of the quarter-chord point of the airfoil, since that is the reference point used for the moment coefficient. RPY support has been excluded, requiring the axes of the airfoil to line up with those of its parent body.
\item \mcode{<profile>} (required) with attribute \mcode{value} which is one of
  the following strings:  
\begin{itemize}
\item The path to a \mcode{.mat} file that can be loaded and contains the three variables ''CLSpline, CDSpline, CMSpline''
\item The string, 'flat plate'
\item File location of a .dat file generated by Xfoil
\item A NACA airfoil designation: 'NACA0012'
\end{itemize}
\item \mcode{<chord>} (required) with attribute \mcode{value} which is the chord
  length in meters.  
\item \mcode{<span>} (required) with attribute \mcode{value} which is the span of
  the wing in meters.  
\item \mcode{<stall_angle>} (required) with attribute \mcode{value} which is the
  angle in degrees upon which the lift and drag performance returns to
  that of a flat plate.  (this value is ignored if the profile is set
  to a flat plate).  
\item \mcode{<nominal_speed>} (required) with attribute \mcode{value} which is an
  approximate nominal speed in meters per second used to calculate the
  Reynolds number around which we design the aerodynamic coefficients.
\end{itemize}
\item Elements for type \mcode{Thrust}
\begin{itemize}
\item Thrust elements produce a force on a point on the parent body in a specified direction.  The magnitude of the force is scaled from an input to the system. Possible applications are the throttle input and force produced by a propeller on an airplane, or thrusters on the hands and feet of an Atlas robot to make Iron Man.
\item \mcode{<parent>} (required) with attribute \mcode{link}. String naming the link on which the force will be applied.
\item \mcode{<origin>} (required) with attribute \mcode{xyz}. String with three numbers designating the point (in link coordinates) where the force will be applied.
\item \mcode{<direction>} (required) with attribute \mcode{xyz}.  A \emph{unit} vector specifying the direction of the force, defined in the xyz axes of the parent link.
\item \mcode{scaleFactor} (required) with attribute \mcode{value} which scales the dimensionless input to Newtons of force.
\item \mcode{limits} (optional) with attribute \mcode{value}. High and low limits of the input, before scaleFactor is applied.  Defaults to -Inf and Inf if omitted.
\end{itemize}

\end{itemize}

\end{itemize}
