
\chapter{Universal Robot Description Format (URDF)}

This appendix documents our support for the Universal Robot Description Format (URDF).

The primary documentation for the URDF specification is available on the ROS website:\\
\centerline{ \url{http://www.ros.org/wiki/urdf/XML}}

\smallskip \noindent
Drake supports most of the tags specified in this documentation, including the transmission elements which are documented seperately here\\ \centerline{\url{http://ros.org/wiki/urdf/XML/Transmission}}  

\smallskip
However, there is currently little or no support for the \mcode{<sensor>} elements.  We have added a few additional tags

\begin{itemize}
\item \mcode{<loop_joint>}
\begin{itemize}
\item Adds support for simple closed-loop kinematic chains.  For an example, see \mcode{examples/SimpleFourBar/FourBar.urdf}.  
\item Attributes
\begin{itemize}
\item \mcode{name} (required)
\item \mcode{type} (required). Currently must be 'continuous'.  The intent is to eventually support all of the same types as a \mcode{<joint>}. See the \mcode{<joint>} element documentation. 
\end{itemize}
\item Elements
\begin{itemize}
\item \mcode{<link1>} (required)
\begin{itemize}
\item Attributes: \mcode{link} (required). String naming the link on one side of the joint.
\item Elements: \mcode{<origin>} (optional). Location (and orientation) of joint relative to the coordinate system on link 1. Default is \mcode{xyz=[0 0 0]}, \mcode{rpy=[0 0 0]}. 
\end{itemize}
\item \mcode{<link2>} (required)
\begin{itemize}
\item Attributes: \mcode{link} (required)
\item Elements: \mcode{<origin>} (optional). Location (and orientation) of joint relative to the coordinate system on link 2. 
\end{itemize}
\item \mcode{<axis>} (optional) 
\item Note: We do not support dynamics or torque at the \mcode{<loop_joint>} yet. (see bug 921) 
\end{itemize}
\end{itemize}

\item \mcode{<force_element>}
\begin{itemize}
\item Adds dynamic, force-producing elements like springs and
  aerodynamics
\item Attributes
\begin{itemize}
\item \mcode{name} (required).
\item \mcode{type} (required).  Currently must be
  \mcode{LinearSpringDamper}.  More types will be added soon.
\end{itemize}
\item Elements for type \mcode{LinearSpringDamper}
\begin{itemize}
\item \mcode{<link1>} (required)  
\begin{itemize}
\item Attributes: \mcode{link} (required). String naming the link on
  one side of the joint; \mcode{<xyz>} (optional). Location of joint relative to the coordinate system on link 1. Default is \mcode{xyz=[0 0 0]}.
\end{itemize}
\item \mcode{<link2>} (required)
\begin{itemize}
\item Attributes: \mcode{link} (required). String naming the link on
  one side of the joint; \mcode{<xyz>} (optional). Location of joint relative to the coordinate system on link 1. Default is \mcode{xyz=[0 0 0]}.
\end{itemize}
\item \mcode{<rest_length>} (default 0)
\item \mcode{<stiffness>} in Newtons per meter (default 0)
\item \mcode{<damping>} in Newton seconds per meter.  positive values
  resist motion.  (default 0)
\end{itemize}
\item Elements for type \mcode{Wing}
\begin{itemize}
\item \mcode{<parent>} (required). 
\begin{itemize} 
\item Attributes: \mcode{<link>} (required).  String naming
  the link on which this wing is attached.
\end{itemize}
\item \mcode{<origin>} with attributes \mcode{xyz} and \mcode{rpy}.
  This defines the position of the aerodynamic center, and the
  orientation of the wing relative to the parent link.
\item \mcode{<profile>} (required) with attribute \mcode{value} which is one of
  the following strings:  
\item \mcode{<chord>} (required) with attribute \mcode{value} which is the chord
  length in meters.  
\item \mcode{<span>} (required) with attribute \mcode{value} which is the span of
  the wing in meters.  
\item \mcode{<stall_angle>} (required) with attribute \mcode{value} which is the
  angle in degrees upon which the lift and drag performance returns to
  that of a flat plate.  (this value is ignored if the profile is set
  to a flat plate).  
\item \mcode{<nominal_speed>} (required) with attribute \mcode{value} which is an
  approximate nominal speed in meters per second used to calculate the
  Reynolds number around which we design the aerodynamic coefficients.
\end{itemize}
\end{itemize}

\end{itemize}
