\tolerance=10000
\documentclass{article}
\usepackage{amsmath,times,graphicx,url}

\newcommand{\pd}[2]{\frac{\partial #1}{\partial #2}}
\newcommand{\bq}{{\bf q}}
\newcommand{\bx}{{\bf x}}
\newcommand{\bu}{{\bf u}}
\newcommand{\bw}{{\bf w}}
\newcommand{\bz}{{\bf z}}
\newcommand{\bK}{{\bf K}}
\newcommand{\balpha}{{\bf \alpha}}
\newcommand{\bbeta}{{\bf \beta}}
\newcommand{\avg}[1]{E\left[ #1 \right]}
\DeclareMathOperator*{\find}{find}
\newcommand{\subjto}{\text{subject to}}
\DeclareMathOperator*{\argmin}{argmin}
\DeclareMathOperator*{\sgn}{sgn}

\usepackage{xcolor}
% switch to other comment command to remove all comments.
\newcommand{\mycomment}[2]{\textcolor{red}{#1}\footnote{\textcolor{red}{#2}}}
%\newcommand{\mycomment}[1]{#1}


\title{Drake Implementation of Rigid Body Dynamics Algorithms with Constraints}

\begin{document}
\maketitle

More complete write-up may follow.  For now, just some quick notes.


\section{Bilateral Position Constraints}

Consider constraint equation $$\phi(q) = 0.$$  These arise, for
instance, when the system has a closed kinematic chain.  
The (floating base) equations of motion can be written 
$$H(q)\ddot{q} + C(q,\dot{q}) = B u + J(q)^T \lambda,$$
where $J(q) = \pd{\phi}{q}$ and $\lambda$ is the constraint force.

To solve for $\lambda$, observe that when the constraint is imposed, 
$\phi(q)=0$ and therefor $\dot{\phi}=0$ and $\ddot{\phi}=0$.  Writing
this out, we have \begin{gather*}\dot{\phi} = J(q)\dot{q}=0,\\
  \ddot{\phi} = J(q) \ddot{q} + \dot{J}(q) \dot{q} = 0.\end{gather*}
Inserting the dynamics and solving for $\lambda$ yields $$\lambda =- (J
H^{-1} J^T)^+ (J H^{-1} (Bu - C) + \dot{J}\dot{q}).$$ The
$^+$ notation refers to a Moore-Penrose pseudo-inverse.  In most cases, there
are less constraints than degrees of freedom, in which case the
inverse has a unique solution (and the traditional inverse could have
been used).  But the pseudo-inverse also works in cases where the
system is over-constrained. 

For numerical stability, I would like to add a restoring force to
this constraint in the event that the constraint is not satisfied to
numerical precision.  To accomplish this, I'll ask for $$\ddot\phi =
-\frac{2}{\epsilon} \dot\phi(q) - \frac{1}{\epsilon^2} \phi(q).$$
Carrying this through
yields 
\begin{gather*}
\lambda =- (J
H^{-1} J^T)^+ (J H^{-1} (Bu - C) + (\dot{J} + \frac{2}{\epsilon}
J)\dot{q} + \frac{1}{\epsilon^2} \phi).
\end{gather*}


\section{Bilateral Velocity Constraints}

Consider the constraint equation 
$$\psi(q,\dot{q}) = 0,$$
where $\pd{\psi}{\dot{q}} =\ne 0.$  These are less common, but arise
when, for instance, a joint is driven through a prescribed motion.  
Here, the manipulator equations are given by $$H(q)\ddot{q} + C = B u
+ \pd{\psi}{\dot{q}}^T \lambda.$$
To solve for $\lambda$, we take $$\dot{\psi} = \pd{\psi}{q} \dot{q} +
\pd{\psi}{\dot{q}}\ddot{q} = 0,$$ which yields
$$\lambda = - \left( \pd{\psi}{\dot{q}} H^{-1} \pd{\psi}{\dot{q}}
  \right)^+ \left[ \pd{\psi}{\dot{q}} H^{-1} (Bu - C) + \pd{\psi}{q}
    \dot{q} \right].$$
 
Again, for numerical stability, we as instead for $\dot{\psi} =
-\frac{1}{\epsilon} \psi$, which yields
$$\lambda = - \left( \pd{\psi}{\dot{q}} H^{-1} \pd{\psi}{\dot{q}}
  \right)^+ \left[ \pd{\psi}{\dot{q}} H^{-1} (Bu - C) + \pd{\psi}{q}
    \dot{q} + \frac{1}{\epsilon} \psi \right].$$

\section{Unilateral Position Constraints}

Consider the constraint equation $$\phi(q) \ge 0.$$ One common example
of this, for instance, is a joint limit.  The dynamics of unilateral
constraints contain to pieces: the continuous dynamics when the
constraint is inactive ($\phi(q) > 0$) or active ($\phi(q) = 0$), but
also an impulsive event when the constraint becomes active
($\phi(q(t))=0, \phi(q(t-\epsilon))>0$).  We model this as a hybrid
transition.  There is no corresponding
event when the constraint transitions to inactive.  

\subsection{Continuous Dynamics}

The continuous equations are governed by $$H\ddot{q} + C = B u + J^T
\lambda,$$ where $J = \pd{\phi}{q}$.  
Let us consider the solution for different cases.  
\begin{itemize}
\item If $\phi > 0$ the constraint is inactive, and $\lambda = 0$.  
\item Otherwise $\phi = 0$, and 
\begin{itemize}
\item if $\dot{\phi} > 0$, then the constraint is going inactive, and
  $\lambda = 0$.  
\item otherwise $\dot{\phi} = 0$, and 
\begin{itemize}
\item $\ddot{\phi} > 0$, and $\lambda = 0$
\item or $\ddot{\phi} = 0$, and $\lambda > 0$.  
\end{itemize}
\end{itemize}
\end{itemize}
For the case when $\ddot{\phi} = 0, \lambda > 0$, we have (as in the
bilateral position constraints) $$\lambda =- (J
H^{-1} J^T)^+ (J H^{-1} (Bu - C) + \dot{J}\dot{q}).$$

As a result, if $\phi>0$ or $\dot{\phi}$, then $\lambda = 0$.  Otherwise, we have
to solve for $\ddot{q}$ and $\lambda$ simultaneously to determine
which constraints are active.  We can accomplish this by solving a linear complementarity problem (LCP):
\begin{eqnarray*}
  \find & & \ddot{\phi}, \lambda \\
 \subjto & & \ddot{\phi} \ge 0, \lambda \ge 0, \\
& & \ddot{\phi} = \dot{J} \dot{q} + J H^{-1} (Bu - C + J^T \lambda), \\
 & & \forall_i ~ \ddot\phi_i \lambda_i = 0.
\end{eqnarray*}
Then $\ddot{q}$ follows from $\ddot{q} = H^{-1}( Bu - C + J^T \lambda ).$

For numerical stability, I must also consider when $\phi < 0$ and/or
$\phi=0,\dot{\phi}<0$, so I instead ask for $$\ddot{\phi} \ge
-\frac{2}{\epsilon} \dot{\phi} - \frac{1}{\epsilon^2} \phi,$$ given
the conditions
\begin{itemize} 
\item If $\phi>0$ or $\dot{\phi} > -\frac{1}{\epsilon} \phi$, then
  $\lambda = 0$.
\item Otherwise, take $\alpha = \ddot\phi + \frac{2}{\epsilon}\dot\phi
  + \frac{1}{\epsilon^2} \phi$ to write
\begin{eqnarray*}
  \find & & \alpha, \lambda \\
 \subjto & & \alpha \ge 0, \lambda \ge 0, \\
& & \alpha = \dot{J} \dot{q} + J H^{-1} (Bu - C + J^T \lambda) -
\frac{2}{\epsilon} \dot\phi - \frac{1}{\epsilon^2} \phi, \\
 & & \forall_i ~ \alpha_i \lambda_i = 0.
\end{eqnarray*}
\end{itemize}

\subsection{Impulsive Event}

The collision event is described by the
zero-crossings (from positivie to negative) of the scalar function
$\phi(q)$, and that after the impact we impose the constraint that
$\phi = 0$.  Using $$H\ddot{q} + C = Bu + J^T \lambda,$$ 
$\lambda$ is now an impulsive force that 
well-defined when integrated over the time of the collision (denoted
$t_c^-$ to $t_c^+$).  Integrate both sides of the equation
over that (instantaneous) interval:
\begin{align*}\int_{t_c^-}^{t_c^+} dt\left[  H \ddot{q} + C \right] =
\int_{t_c^-}^{t_c^+} dt \left[ Bu + J^T \lambda \right]
\end{align*}
Since $q$ and $u$ are constants over this interval, we are left
with
$$H\dot{q}^+ - H\dot{q}^- = J^T \int_{t_c^-}^{t_c^+} \lambda dt,$$ 
where $\dot{q}^+$ is short-hand
for $\dot{q}(t_c^+)$.  Multiplying both sides by $J H^{-1}$, we have 
$$ J \dot{q}^+ - J\dot{q}^- = J H^{-1} J^T \int_{t_c^-}^{t_c^+} \lambda dt.$$
But the first term on the left is zero because after the collision,
$\dot\phi = 0$, yielding:
$$\int_{t_c^-}^{t_c^+} \lambda dt = - \left[ J H^{-1} J^T \right]^+
J \dot{q}^-.$$  Substituting this back in above results in 
$$\dot{q}^+ = \left[ I - H^{-1} J^T \left[ J H^{-1} J^T \right]^+ J
\right] \dot{q}^-.$$

So far, I assume all collisions are inelastic.  Elastic
collisions are also possible, but require some care implementing
restitution (especially in the contact case described below).  An
excellent discussion can be found in chapter 3 of \cite{Mirtich96}.  


\section{Contact Constraints}

\subsection{Continuous Dynamics}

\subsection{Impulsive Event}


\section{Putting it all together}


\bibliographystyle{plain}
\bibliography{elib}

\end{document}
